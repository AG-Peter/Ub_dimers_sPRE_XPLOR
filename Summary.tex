%%%%%%%%%%%%%%%%%%%%%%%%%%%%%%%%%%%%%%%%%
% Beamer Presentation
% LaTeX Template
% Version 1.0 (10/11/12)
%
% This template has been downloaded from:
% http://www.LaTeXTemplates.com
%
% License:
% CC BY-NC-SA 3.0 (http://creativecommons.org/licenses/by-nc-sa/3.0/)
%
%%%%%%%%%%%%%%%%%%%%%%%%%%%%%%%%%%%%%%%%%

%----------------------------------------------------------------------------------------
%	PACKAGES AND THEMES
%----------------------------------------------------------------------------------------

\documentclass{beamer}

\mode<presentation> {

% The Beamer class comes with a number of default slide themes
% which change the colors and layouts of slides. Below this is a list
% of all the themes, uncomment each in turn to see what they look like.

%\usetheme{default}
%\usetheme{AnnArbor}
%\usetheme{Antibes}
%\usetheme{Bergen}
%\usetheme{Berkeley}
%\usetheme{Berlin}
%\usetheme{Boadilla}
%\usetheme{CambridgeUS}
%\usetheme{Copenhagen}
%\usetheme{Darmstadt}
%\usetheme{Dresden}
%\usetheme{Frankfurt}
%\usetheme{Goettingen}
%\usetheme{Hannover}
%\usetheme{Ilmenau}
%\usetheme{JuanLesPins}
%\usetheme{Luebeck}
%\usetheme{Madrid}
%\usetheme{Malmoe}
%\usetheme{Marburg}
%\usetheme{Montpellier}
%\usetheme{PaloAlto}
%\usetheme{Pittsburgh}
%\usetheme{Rochester}
%\usetheme{Singapore}
%\usetheme{Szeged}
%\usetheme{Warsaw}

% As well as themes, the Beamer class has a number of color themes
% for any slide theme. Uncomment each of these in turn to see how it
% changes the colors of your current slide theme.

%\usecolortheme{albatross}
%\usecolortheme{beaver}
%\usecolortheme{beetle}
%\usecolortheme{crane}
%\usecolortheme{dolphin}
\usecolortheme{dove}
%\usecolortheme{fly}
%\usecolortheme{lily}
%\usecolortheme{orchid}
%\usecolortheme{rose}
%\usecolortheme{seagull}
%\usecolortheme{seahorse}
%\usecolortheme{whale}
%\usecolortheme{wolverine}

%\setbeamertemplate{footline} % To remove the footer line in all slides uncomment this line
%\setbeamertemplate{footline}[page number] % To replace the footer line in all slides with a simple slide count uncomment this line

%\setbeamertemplate{navigation symbols}{} % To remove the navigation symbols from the bottom of all slides uncomment this line
}

\usepackage{graphicx} % Allows including images
\usepackage{booktabs}
\usepackage{biblatex} % Allows the use of \toprule, \midrule and \bottomrule in tables
\usepackage{subcaption}
\usepackage{csquotes}
\usepackage{url}

%----------------------------------------------------------------------------------------
%	TITLE PAGE
%----------------------------------------------------------------------------------------

\title[Intersection of MD and NMR]{Intersection of multiscale MD and NMR} % The short title appears at the bottom of every slide, the full title is only on the title page

\author{Kevin Sawade} % Your name
\institute[Universit\"at Konstanz] % Your institution as it will appear on the bottom of every slide, may be shorthand to save space
{
Universit\"at Konstanz \\ % Your institution for the title page
\medskip
\textit{kevin.sawade@uni-konstanz.de} % Your email address
}
\date{\today} % Date, can be changed to a custom date

\begin{document}

\begin{frame}
\titlepage % Print the title page as the first slide
\end{frame}

\begin{frame}
\frametitle{Overview} % Table of contents slide, comment this block out to remove it
\tableofcontents % Throughout your presentation, if you choose to use \section{} and \subsection{} commands, these will automatically be printed on this slide as an overview of your presentation
\end{frame}

%----------------------------------------------------------------------------------------
%	PRESENTATION SLIDES
%----------------------------------------------------------------------------------------

%------------------------------------------------
\section{Data Overview} % Sections can be created in order to organize your presentation into discrete blocks, all sections and subsections are automatically printed in the table of contents as an overview of the talk
%------------------------------------------------

\subsection{Available data} % A subsection can be created just before a set of slides with a common theme to further break down your presentation into chunks

\begin{frame}
\frametitle{Data available from previous publications}
\begin{enumerate}
    \item All-atom MD simulations using an altered GROMOS54a7 forcefield. Started from extended conformations.
    \item Coarse-grained MD simulations using MARTINI v2.2 ff.
    \item All-atom MD simulations started from the 4 lowest sketch-map basins of the CG map using BACKWARD.
    \item All-atom MD simulations started from 10 random points around the 4 lowest sketch-map basins of the CG map using BACKWARD.
    \item All-atom MD simulations started from $\mathrm{\chi_3}$-rotamers of extended structures.
\end{enumerate}
\end{frame}

%------------------------------------------------

\begin{frame}
\frametitle{Overview}
\begin{figure}
\begin{subfigure}{.25\textwidth}
\centering
\includegraphics[width=\linewidth]{/home/kevin/projects/tobias_schneider/new_images/summary_movie.pdf}
\end{subfigure}%
\begin{subfigure}{.75\textwidth}
\centering
\includegraphics[width=\linewidth]{/home/kevin/projects/tobias_schneider/new_images/summary_k6_simulation.pdf}
\end{subfigure}
\end{figure}
\end{frame}

%------------------------------------------------

\section{Methods and Results}

\subsection{Analysis Scheme}

\begin{frame}
\frametitle{Analysis Steps}
\begin{enumerate}
\item Extract high-dimensional CVs from all simulation frames.
\end{enumerate}

\begin{figure}
    \includegraphics[width=\textwidth]{/home/kevin/projects/tobias_schneider/new_images/summary_RWMD.pdf}
\end{figure}

%\begin{enumerate}
%  \setcounter{enumi}{4}
%  \item fifth element
%\end{enumerate}

\end{frame}

%------------------------------------------------

\begin{frame}
\frametitle{Analysis Steps}
\begin{enumerate}
  \setcounter{enumi}{1}
  \item Run EncoderMap with the high-dimensional CVs as input data.
\end{enumerate}
    \begin{figure}
    \includegraphics[width=\textwidth]{/home/kevin/projects/tobias_schneider/new_images/summary_encodermap.pdf}
\end{figure}
\end{frame}

%------------------------------------------------

\begin{frame}
\frametitle{Analysis Steps}
\begin{enumerate}
  \setcounter{enumi}{2}
  \item Cluster data using HDBSCAN.
\end{enumerate}
    \begin{figure}
    \includegraphics[width=\textwidth]{/home/kevin/projects/tobias_schneider/new_images/summary_hdbscan.pdf}
\end{figure}
\end{frame}

%------------------------------------------------

\begin{frame}
\frametitle{Analysis Steps}
\begin{enumerate}
  \setcounter{enumi}{2}
  \item Write a code library, that poses as an interface between current python packages and XPLOR. (\url{https://github.com/kevinsawade/xplor_functions})
  \item Identify possible settings/arguments and define sensible defaults (solution concentration, probe radius, etc.)
  \item Parallelize the scoring for faster throughput.
  \item Manually parse .psf files to include 15N relaxation data.
\end{enumerate}

\end{frame}

%------------------------------------------------

\begin{frame}
\frametitle{Analysis Steps}

\begin{figure}
\includegraphics[width=\textwidth]{/home/kevin/projects/tobias_schneider/new_images/summary_code.pdf}
\end{figure}


\end{frame}

%------------------------------------------------

\begin{frame}
\frametitle{Analysis Steps}
\begin{enumerate}
  \setcounter{enumi}{7}
  \item Normalize sPRE computations.
\end{enumerate}

\begin{itemize}
    \item Don't consider the fast-exchanging residues.
    \item Consider proximal and distal unit separately.
    \item From all simulation frames calculate the variance of the sPRE values for this residue.
    \item Take the 10 (not fast exchanging) residues with the smallest variances.
    \item Calculate the factor $f_i$ from $f_i = \frac{v_{i, exp}}{v_{i, sim}}$ for every of these 10 residues.
    \item Calculate the mean of these ten factors as $F = \frac{\sum f_i}{N}$
    \item Use $F$ as a factor to normalize the sPRE values of the proximal unit.
\end{itemize}

\end{frame}

%------------------------------------------------

\begin{frame}
\frametitle{Analysis Steps}
\begin{enumerate}
  \setcounter{enumi}{8}
  \item Calculate sPRE for AA conformations.
\end{enumerate}
    \begin{figure}
    \includegraphics[width=\textwidth]{/home/kevin/projects/tobias_schneider/new_images/summary_spre.pdf}
\end{figure}

\end{frame}

%------------------------------------------------

\begin{frame}
\frametitle{Analysis Steps}
\begin{enumerate}
  \setcounter{enumi}{9}
  \item Calculate 15N relaxations for AA conformations.
\end{enumerate}
    \begin{figure}
    \includegraphics[width=\textwidth]{/home/kevin/projects/tobias_schneider/new_images/summary_15N_600.pdf}
\end{figure}

\end{frame}

%------------------------------------------------

\subsection{Cluster Analysis}
\begin{frame}
\frametitle{Cluster coefficients}
Solve:
\begin{equation*}
\begin{bmatrix}
   v_{exp, MET1} \\
   v_{exp, GLN2} \\
   \vdots \\
   v_{exp, GLY76}
\end{bmatrix}
    = x_1 \cdot \begin{bmatrix}
   v_{clu_1, MET1} \\
   v_{clu_1, GLN2} \\
   \vdots \\
   v_{clu_1, GLY76}
\end{bmatrix}
    + x_2 \cdot \begin{bmatrix}
   v_{clu_2, MET1} \\
   v_{clu_2, GLN2} \\
   \vdots \\
   v_{clu_2, GLY76}
\end{bmatrix}
    + \ldots + x_n \cdot \begin{bmatrix}
   v_{clu_n, MET1} \\
   v_{clu_n, GLN2} \\
   \vdots \\
   v_{clu_n, GLY76}
\end{bmatrix}
\end{equation*}

    for $(x_1, x_2, ..., x_n) \in \mathbb{N}$

\end{frame}

%------------------------------------------------

\begin{frame}
\frametitle{Single cluster}
A cluster is extracted from the complete (CG and AA) ensemble but does only contain AA conformations.
\begin{itemize}
\item Its contribution to the whole ensemble.
\item A coefficient from the linear combination of clusters.
\end{itemize}
\begin{figure}
\includegraphics[width=\textwidth]{/home/kevin/projects/tobias_schneider/new_images/summary_single_cluster.png}
\end{figure}

\end{frame}

%------------------------------------------------

\begin{frame}
\frametitle{Is the best fitting structure in a cluster?}
\begin{figure}
\includegraphics[width=\textwidth]{/mnt/data/kevin/xplor_analysis_files/best_fitting_in_cluster.png}
\end{figure}

\end{frame}

%------------------------------------------------

\begin{frame}
\frametitle{RMSD matrices for clusters}

\end{frame}

%------------------------------------------------

\begin{frame}
\frametitle{Multiple clusters}
Multiple clusters better represent the nature of the sPRE ensemble. A large coefficient in the linear combination does not necessarily mean that this cluster has similar sPRE values as the experiment.

\begin{figure}
\includegraphics[width=\textwidth]{/home/kevin/projects/tobias_schneider/new_images/summary_quality_factors.pdf}
\end{figure}

\end{frame}

%------------------------------------------------

\begin{frame}
\frametitle{There's always one low-score outlier. Good cluster?}
Multiple clusters better represent the nature of the sPRE ensemble. A large coefficient in the linear combination does not necessarily mean that this cluster has similar sPRE values as the experiment.

\end{frame}

%------------------------------------------------

\subsection{General shape of the diUbi proteins}
\begin{frame}
\frametitle{Test Tensors of inertia}

\begin{figure}
\includegraphics[width=\textwidth]{/mnt/data/kevin/xplor_analysis_files/inertia_distribution_k6.png}
\end{figure}


\end{frame}

%------------------------------------------------

\begin{frame}
\frametitle{Render of a structrues within a I\_xx gaussian}

\end{frame}

%------------------------------------------------

\begin{frame}
\frametitle{Surface coverage might also be a possibility.}

\begin{figure}
\includegraphics[width=\textwidth]{/home/kevin/projects/tobias_schneider/new_images/summary_surface_coverage.pdf}
\end{figure}

\end{frame}

%------------------------------------------------

\begin{frame}
\frametitle{Better: pseudo-dihedral and cog-distance}
    \begin{figure}
    \includegraphics[width=\textwidth]{/home/kevin/projects/tobias_schneider/new_images/summary_pseudo_torsions.pdf}
\end{figure}

\end{frame}

%------------------------------------------------

\begin{frame}
\frametitle{Blocks of Highlighted Text}
\begin{block}{Block 1}
Lorem ipsum dolor sit amet, consectetur adipiscing elit. Integer lectus nisl, ultricies in feugiat rutrum, porttitor sit amet augue. Aliquam ut tortor mauris. Sed volutpat ante purus, quis accumsan dolor.
\end{block}

\begin{block}{Block 2}
Pellentesque sed tellus purus. Class aptent taciti sociosqu ad litora torquent per conubia nostra, per inceptos himenaeos. Vestibulum quis magna at risus dictum tempor eu vitae velit.
\end{block}

\begin{block}{Block 3}
Suspendisse tincidunt sagittis gravida. Curabitur condimentum, enim sed venenatis rutrum, ipsum neque consectetur orci, sed blandit justo nisi ac lacus.
\end{block}
\end{frame}

%------------------------------------------------

\begin{frame}
\frametitle{Multiple Columns}
\begin{columns}[c] % The "c" option specifies centered vertical alignment while the "t" option is used for top vertical alignment

\column{.45\textwidth} % Left column and width
\textbf{Heading}
\begin{enumerate}
\item Statement
\item Explanation
\item Example
\end{enumerate}

\column{.5\textwidth} % Right column and width
Lorem ipsum dolor sit amet, consectetur adipiscing elit. Integer lectus nisl, ultricies in feugiat rutrum, porttitor sit amet augue. Aliquam ut tortor mauris. Sed volutpat ante purus, quis accumsan dolor.

\end{columns}
\end{frame}

%------------------------------------------------
\section{Second Section}
%------------------------------------------------

\begin{frame}
\frametitle{Table}
\begin{table}
\begin{tabular}{l l l}
\toprule
\textbf{Treatments} & \textbf{Response 1} & \textbf{Response 2}\\
\midrule
Treatment 1 & 0.0003262 & 0.562 \\
Treatment 2 & 0.0015681 & 0.910 \\
Treatment 3 & 0.0009271 & 0.296 \\
\bottomrule
\end{tabular}
\caption{Table caption}
\end{table}
\end{frame}

%------------------------------------------------

\begin{frame}
\frametitle{Theorem}
\begin{theorem}[Mass--energy equivalence]
$E = mc^2$
\end{theorem}
\end{frame}

%------------------------------------------------

\begin{frame}[fragile] % Need to use the fragile option when verbatim is used in the slide
\frametitle{Verbatim}
\begin{example}[Theorem Slide Code]
\begin{verbatim}
\begin{frame}
\frametitle{Theorem}
\begin{theorem}[Mass--energy equivalence]
$E = mc^2$
\end{theorem}
\end{frame}\end{verbatim}
\end{example}
\end{frame}

%------------------------------------------------

\begin{frame}
\frametitle{Figure}
Uncomment the code on this slide to include your own image from the same directory as the template .TeX file.
%\begin{figure}
%\includegraphics[width=0.8\linewidth]{test}
%\end{figure}
\end{frame}

%------------------------------------------------

\begin{frame}[fragile] % Need to use the fragile option when verbatim is used in the slide
\frametitle{Citation}
An example of the \verb|\cite| command to cite within the presentation:\\~

This statement requires citation \cite{p1}.
\end{frame}

%------------------------------------------------

\begin{frame}
\frametitle{References}
\footnotesize{
%\begin{thebibliography}{99} % Beamer does not support BibTeX so references must be inserted manually as below
%\bibitem[Smith, 2012]{p1} John Smith (2012)
%\newblock Title of the publication
%\newblock \emph{Journal Name} 12(3), 45 -- 678.
%\end{thebibliography}
\bibliography{mybib.bib}
\printbibliography
}
\end{frame}

%------------------------------------------------

\begin{frame}
\Huge{\centerline{The End}}
\end{frame}

%----------------------------------------------------------------------------------------

\end{document}

%% Preamble
%\documentclass[11pt]{article}
%
%% Packages
%\usepackage{amsmath}
%\usepackage{graphicx}
%\usepackage{here}
%
%% Document
%\begin{document}
%
%\section{Summary for sPRE calculations with XPLOR}
%
%    The summary of this project comprises three main points: Results, Problems, Methods
%
%    \subsection{Results}
%
%    \paragraph{Running XPLOR on MD data} Th
%
%    \paragraph{How many clusters are needed?} This question can be answered via this graph:
%
%    \begin{figure}[H]
%        \includegraphics{/mnt/data/kevin/xplor_analysis_files/quality_factors_k6}
%    \end{figure}
%
%    On the x-axis is the number of considered clusters. This includes all permustations,  the 2 considered clusters are
%    found by permuting over the number of clusters: (0, 1), (0, 2), ... (0, 12), (1, 1), ... (1, 2), ...
%    The y-axis is the mean abs difference between the calucaltion and the linear combination of the clusters. The
%    linear combination was built as such:
%
%    \begin{align}
%    \mathbf{sPRE} &= v_m &= x1 \cdot v_{c1} + x2 \cdot v_{c2} \hdots \\
%    \mathbf{sPRE} &= y &= \begin{bmatrix}
%           v_(m, a) \\
%            \\
%           \vdots \\
%           x_{m}
%         \end{bmatrix}
%    (sPRE_1, sPRE_2, ...) &= x1 * (sPRE_1, sPRE_2, sPRE_3), x2 * (sPRE_1, sPRE_2, ...)
%    \end{align}
%
%
%\end{document}