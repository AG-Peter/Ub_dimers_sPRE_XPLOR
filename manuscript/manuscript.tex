Skip to content
Features & Benefits
Templates
Produkte und Preise
Hilfe
Registrieren
Anmelden
Share your thoughts on the Overleaf Template Gallery!
×Schließen
Template for submissions to Wiley StatsRefOfficial
AuthorWiley
LicenseOther (as stated in the work)
Abstract
This is the LaTeX template for Wiley StatsRef. Please refer to the journal’s author guidelines in order to confirm your manuscript adheres to the journal’s requirements for submissions.

Once your manuscript is complete, simply use the "Submit to Journal" option in the Overleaf editor to submit your files directly to the journal for processing.

If you're new to LaTeX, check out our free online introduction to help you get started, or please get in touch if you have any questions.

Tags
Find More Templates
Template for submissions to Wiley StatsRef
×
Quelldateien
\documentclass[cmbright]{WileySTAT-V1} %% [cmbright,doublespace], [demo] for draft mode

%\usepackage{mathtime}   %% Uses times
\usepackage[bookmarksopen,bookmarksnumbered,citecolor=blue,urlcolor=blue]{hyperref} %% Online requirements
\usepackage{natbib}     %% Used for Author Name and year

\usepackage{dcolumn}    %% For Tabular decimal alignments
\newcolumntype{d}[1]{D{.}{.}{#1}}

%% Journal informations \ref{}
\def\volumeyear{2018}
\def\volumenumber{99}
\def\DOI{ISBN.stat00999.pub9}

%% Set running heads
\runninghead{First Author and Second Author}{Short title}

%% AMS Theorems
\newtheorem{theorem}{Theorem}
\newtheorem{proposition}[theorem]{Proposition} %% remove [theorem] for Unique numbering
\theoremstyle{plain}
\newtheorem{example}[theorem]{Example} %% remove [theorem] for Unique numbering
\newtheorem{definition}[theorem]{Definition} %% remove [theorem] for Unique numbering

\raggedbottom

\begin{document}

\title{Wiley \LaTeX2e\ Authoring Template for STAT~Journal}

\author{FirstName1 Surname1\affil{a}\corrauth and
        FirstName2 Surname2\affil{b}}

\address{%
\affilnum{a}First authors's address \\
\affilnum{b}Second author's address}

\articlenote{This is article note. Processes play a keyrole in the kinetics of many microstructural changes that occur during processing of metals, alloys, ceramics, semiconductors, glasses, and polymers.}
\corremail{an@firstauthor}

\begin{abstract}
This paper describes the use of \LaTeXe\ with \textbf{WileySTAT-V1} class file for setting papers for \textit{Stat}.
\end{abstract}

\keywords{Keywords; \LaTeXe; \textit{Stat}}

\maketitle

\section{Introduction}\label{sec1}

Many authors submitting to research journals use \LaTeXe\ to prepare their papers. This paper describes the \textbf{WileySTAT-V1} class file which can be used to convert articles produced with other \LaTeXe\ class files into the correct form for publication in \textit{Stat}.

The \textbf{WileySTAT-V1} class file preserves much of the standard \LaTeXe\ interface so that any document which was produced using the standard \LaTeXe\ article style can easily be converted to work with the \textbf{WileySTAT-V1} style. However, the width of text and typesize will vary from that of \textbf{article.cls}; therefore, \emph{line breaks will change} and it is likely that displayed mathematics and tabular material will need re-setting.

In the following sections we describe how to lay out your code to use \textbf{WileySTAT-V1} class file to reproduce the typographical look of \textit{Stat}. However, this paper is not a guide to using \LaTeXe\ and we would refer you to any of the many books available \citep[see, for example,][]{R1,R2,R3}.

\section{Getting Started}\label{sec2}

The \textbf{WileySTAT-V1} class file should run on any standard \LaTeXe\ installation. If any of the fonts, class files or packages it requires are missing from your installation, they can be found on the \emph{\TeX\ Live} or from CTAN.

\section{Article Usage}\label{sec3}

\textit{Stat} is published using proprietary fonts. A reasonable match can be achieved by using the \verb"cmbright" option~as \verb"\documentclass[cmbright]{WileySTAT-V1}". If for any reason you have a problem using the CM~Bright fonts you can easily resort to Computer Modern fonts by removing the \verb"cmbright" option.

Use \verb"doublespace" option as \verb"\documentclass[cmbright,doublespace]{WileySTAT-V1}" for extra line spread.

\begin{enumerate}
\item[(i)] In \verb"\runninghead" it should contain the name of author/s (verso) and short title (recto). If there are three or more authors, list down the surname of first author (with initial) followed by `\emph{et~al.}'.

\item[(ii)] Authors and Addresses: In the \verb"\author{}" tag, put your names in full, the firstname should not be an initial. For authors with affiliations within the US, please indicate your complete address including the state and zip code.

\item[(iii)] Note the use of \verb"\affil" and \verb"\affilnum" to link names and addresses. For authors that share in the same address, they should have the same link letter.

\item[(iv)] The author for correspondence (via email) is marked by \verb"\corrauth" and \verb"\corremail". The latter is used to give the corresponding author's email, to be printed as a footnote prefaced by `Email:'.

\item[(v)] The abstract should be capable of standing by itself, in the absence of the body of the article and of the bibliography. Therefore, it must not contain any reference citations.

\item[(vi)] Keywords are separated by semicolons.
\end{enumerate}

%% Heading 1
\section{Heading 1: Coding text content}\label{sec4}
%% For unnumber Heading -- \section*{Heading 1: Coding text content}
%% For shorting section head use optional -- \section[Heading 1: Coding section]{Heading 1: Coding text content}
This chapter will concentrate on bulk diffusion in solid metals and alloys. Most of the solid elements are metals. Furthermore, diffusion properties and atomic mechanisms of diffusion have most thoroughly been investigated in metallic solids.\footnote{This is footnote text. For anisotropic media and non-cubic crystalline solids $\mathcal{D}$ is a symmetric tensor of rank 2. Each symmetric second rank tensor can be reduced to diagonal form.}

%% Heading 2
\subsection{Heading 2: Coding text content}\label{sec4.1}
%% For unnumber Heading -- \subsection*{Heading 2: Coding text content}
This chapter will concentrate on bulk diffusion in solid metals and alloys. Most of the solid elements are metals. Furthermore, diffusion properties and atomic mechanisms of diffusion have most thoroughly been investigated in metallic solids.

%% Heading 3
\subsubsection{Heading 3: Coding text content}
%% For unnumber Heading -- \subsubsection*{Heading 3: Coding text content}
This chapter will concentrate on bulk diffusion in solid metals and alloys. Most of the solid elements are metals. Furthermore, diffusion properties and atomic mechanisms of diffusion have most thoroughly been investigated in metallic solids.

%% Heading 4
\paragraph{Heading 4: Coding text content.}
This chapter will concentrate on bulk diffusion in solid metals and alloys. Most of the solid elements are metals. Furthermore, diffusion properties and atomic mechanisms of diffusion have most thoroughly been investigated in metallic solids.

\subsection{Unnumbered, Numbered, and Bullet Lists}\label{sec4.2}
Theoretical models which permit the calculation of the composition dependent from the deeper principles using, statistical mechanics are nowadays still not broadly available. Then the strategy illustrated in the previous section of calculating the concentration for certain initial and boundary conditions is not applicable.%% Enumerate
\begin{enumerate} %% Numbered
\item This is Numbered list. The atomic mechanisms of diffusion in crystalline materials are closely connected with defects.
\item This is Numbered list. The atomic mechanisms of diffusion in crystalline materials are closely connected with defects.
\item This is Numbered list. The atomic mechanisms of diffusion in crystalline materials are closely connected with defects.
\end{enumerate}
Theoretical models which permit the calculation of the composition dependent from the deeper principles using, statistical mechanics are nowadays still not broadly available. Then the strategy illustrated in the previous section of calculating the concentration for certain initial and boundary conditions is not applicable.%% Enumerate
\begin{enumerate} %% Unnumbered
\item This is Unnumbered list. The atomic mechanisms of diffusion in crystalline materials are closely connected with defects.
\item This is Unnumbered list. The atomic mechanisms of diffusion in crystalline materials are closely connected with defects.
\item This is Unnumbered list. The atomic mechanisms of diffusion in crystalline materials are closely connected with defects.
\end{enumerate}

This chapter will concentrate on bulk diffusion in solid metals and alloys. Most of the solid elements are metals. Furthermore, diffusion properties and atomic mechanisms of diffusion have most thoroughly been investigated in metallic solids.
\begin{itemize} %% Itemize
\item This is Bullet list. The atomic mechanisms of diffusion in crystalline materials are closely connected with defects.
\item This is Bullet list. The atomic mechanisms of diffusion in crystalline materials are closely connected with defects.
\item This is Bullet list. The atomic mechanisms of diffusion in crystalline materials are closely connected with defects.
\end{itemize}

This chapter will concentrate on bulk diffusion in solid metals and alloys. Most of the solid elements are metals. Furthermore, diffusion properties and atomic mechanisms of diffusion have most thoroughly been investigated in metallic solids.
\begin{enumerate} %% Nested listing
\item[1] This is Nested listing - 1
 \begin{itemize}
 \item This is Nested listing - A
 \item This is Nested listing - B
 \end{itemize}
\item[5] This is Nested listing - 2
 \begin{itemize}
 \item[--] This is Nested listing - A
 \item[--] This is Nested listing - B
 \end{itemize}
\item This is Nested listing - 3
\end{enumerate}

\subsection{Descriptions and quotes}\label{sec4.3}
Typical examples are nucleation of new phases, diffusive phase transformations, precipitation and dissolution of a second phase, recrystallization, high-temperature creep, and thermal oxidation.
%% Description
\begin{description}
\item[First entry] description text description text description text description text description text description text
\item[Second long entry] description text description text description text description text description text description text description text
\item[Third entry] description text description text description text description text description text
\item[Fourth entry] description text description text
\end{description}

This chapter will concentrate on bulk diffusion in solid metals and alloys. Most of the solid elements are metals. Furthermore, diffusion properties and atomic mechanisms of diffusion have most thoroughly been investigated in metallic solids.
\begin{quote}
\textbf{Quote:} The atomic mechanisms of diffusion in crystalline materials are closely connected with diffusion in crystalline materials are closely connected with defects.
\end{quote}
This chapter will concentrate on bulk diffusion in solid metals and alloys. Most of the solid elements are metals. Furthermore, diffusion properties and atomic mechanisms of diffusion have most thoroughly been investigated in metallic solids.


\subsection{Numbered Display Equations}\label{sec4.4}

\textbf{WileySTAT-V1} class file makes the full functionality of \AmS\/\TeX\ available. We encourage the use of the \verb"align", \verb"gather" and \verb"multline" environments for displayed mathematics.

This chapter will concentrate on bulk diffusion in solid metals and alloys. Furthermore, diffusion properties and atomic mechanisms of diffusion have most thoroughly been investigated in metallic solids.
\begin{equation}\label{eq1}
 - \frac{D_1}{V} \left. \frac{\partial C}{\partial x} \right|_{x=0} =
 \begin{cases}
 C_1(y) - \eta_\alpha C_\alpha & \mbox{for $\alpha$ phase} \\
 C_2(y) - \eta_\alpha C_\alpha & \mbox{for $\beta$ phase}
 \end{cases}
\end{equation}
Equation (\ref{eq1}) and Eqs. \ref{eq1}--\ref{eq3} implies that $\mathcal{D}$ varies with direction. In general the diffusion flux and the concentration gradient are not always antiparallel. They are antiparallel for an isotropic medium.
\begin{align}\label{eq2}
 C(x,y) & = B_1\exp \left(-\frac{V}{D}x \right) + \sum_{n=1}^{\infty} B_n \cos \frac{n\pi y}{L} \\
 C(x,y) & = B_2\exp \left\{\left[-\frac{V}{2D} - \left[ \left(\frac{V}{2D} \right)^2 + \left(\frac{n\pi}{L} \right)^2 \right]^{1/2} \right] x \right\}
\end{align}

For anisotropic media and non-cubic crystalline solids $\mathcal{D}$
is a symmetric tensor of rank 2 \cite{R2}. Each symmetric
second rank tensor can be reduced to diagonal form. The diffusion
flux is antiparallel to the concentration gradient only for
diffusion along the orthogonal principal directions. If $x_{1},
x_{2}, x_{3}$ denote these directions and $j_{1}, j_{2}, j_{3}$ the
pertaining components of the diffusion flux, Eq. (\ref{eq2}) can
be written as\begin{multline}\label{eq3}
\left[\mathcal{D} d^{sn} D_{n} \left(\nabla\times\tilde{\mathbf{u}}^{s}\right) +
D_{n}\left(\nabla\times{\mathbf{u}}^{s}\right)-\right]\delta\mathbf{u}^{s}(\mathbf{x},t) \\
= \lim_{\alpha\to0}\big\{ D_{n}\left(\nabla\times\tilde{\mathbf{u}}^{s}\right) +
D_{n}\left(\nabla\times{\mathbf{u}}^{s}\right)-
\left[\nabla\times\left(\mathbf{u}^{s}+\alpha\delta\mathbf{u}^{s}\right)\right]
-D_{n}\left(\nabla\times\tilde{\mathbf{u}}^{s}\right)\left[\nabla\times\mathbf{u}^{s}\right]\big\}/\alpha.
\end{multline}
where diffusion coefficient for a direction $D_{I}, D_{II}, D_{III}$ denote the three principal diffusivities. The diffusion coefficient for a direction $(\alpha_{1}, \alpha_{2}, \alpha_{3})$ is obtained from
\begin{eqnarray}
s(nT_{s}) &= &s(t)\times \sum\limits_{n=0}^{N-1} \delta (t-nT_{s}) \leftrightarrow{\mathrm{DFT}} S \left(\frac{m}{NT_{s}}\right) \nonumber\\
&= &\frac{1}{N} \sum\limits_{n=0}^{N-1} \sum\limits_{k=-N/2}^{N/2-1} s_{k} e^{\mathrm{j}2\pi k\Delta fnT_{s}} e^{-j\frac{2\pi}{N}mn}
\end{eqnarray}

\subsection{Unnumbered Inline and Display Equations}\label{sec4.5}
Although there will be a gradient in the concentration of the trace element, its total concentration can be kept so small that the overall composition of the sample during the investigation does practically not change\footnote{From an atomistic viewpoint this implies that a tracer atom is not influenced by other tracer atoms.}
$$
\frac{\partial c}{\partial t}=\frac{\partial}{\partial
x}\left(\tilde{D}(c)\frac{\partial c}{\partial
x}\right)=\tilde{D}(c)\frac{\partial^{2}c}{\partial
x^{2}}+\frac{\mathrm{d}\tilde{D}(c)}{\mathrm{d}c}\left(\frac{\partial
c}{\partial x}\right)^{2}.
$$
The connection between the macroscopically defined tracer self-diffusion coefficient and the atomistic picture of diffusion is the famous Einstein-Smoluchowski relation discussed in detail in Sect.~\ref{sec4.6}.

Where $l$ denotes the jump length and $\tau$ the mean residence time of an atom on a certain site of the crystal\footnote{Equation (\ref{eq3}) considers only the simplest case: cubic structure, all sites are energetically equivalent, only jumps to nearest neighbours are allowed.}. The quantity $f$ is the correlation factor. For self-diffusion in cubic crystals $f$ is a numeric factor.
\begin{gather*}
U = T \delta S - P \delta V + \sum_{i=1}^c \mu_i \delta N_i, \\
N = \sum_{i=1}^c \mu_i \delta N_i, + \sum_{i=1}^c \mu_i.
\end{gather*}
where diffusion coefficient for a direction $D_{I}, D_{II}, D_{III}$ denote the three principal diffusivities. The diffusion coefficient for a direction $(\alpha_{1}, \alpha_{2}, \alpha_{3})$ is obtained from
\begin{align*}
c_1 &= A_1+c_2A_2c_3A_3 \\
c_1 &= A_1+c_2A_2c_3A_3 \quad \hbox{with multiple lines without number} \\
c_1 &= A_1+c_2A_2c_3A_3 \quad \hbox{with multiple lines without number}
\end{align*}
are obtained. These diffusion coefficients are denoted as impurity diffusion coefficients or sometimes also as foreign atom diffusion coefficients.

\subsection{Figures and Tables}\label{sec4.6}
\textbf{WileySTAT-V1} class file uses the \textbf{graphicx} package for handling figures.

So far we have considered in this section cases where the concentration gradient is the only cause for the flow of matter. We have seen that such situations can be studied using tiny amounts of trace elements in an otherwise homogeneous material. However, from a general viewpoint a diffusion flux is proportional to the gradient of the chemical potential.

\begin{figure} %% Figure 1
\centerline{\includegraphics[height=10pc]{Figure1.eps}}
\caption{A short description of the figure content should go here. Hypothetical data on the effect of Drug A versus
Placebo on the number of study participants with infection present at 1 week, among
participants with Infection $X$.\figsource{This graph was drawn using the
Cochrane Collaboration software, Revman (version 5.3).}\label{F1}}
\end{figure}

For further details on how to size figures, etc., with the \textbf{graphicx} package see, for example, \citet{R1} or \citep{R3}. If figures are available in an acceptable format (for example, .eps, .ps) they will be used but a printed version should always be provided.

\begin{table}[b]%
\caption{This is sample table caption. A researcher must decide which observed variables to include in the theoretical model and how these observed variables measure.\label{tab1}}
{\tabcolsep0pt\begin{tabular*}{\textwidth}{@{\extracolsep{\fill}}lccd{2,3}c@{}}
\toprule
&\multicolumn{2}{@{}c@{}}{\textbf{Spanned heading 1}} & \multicolumn{2}{@{}c@{}}{\textbf{Spanned heading 2}} \\
\cmidrule{2-3}\cmidrule{4-5}
\textbf{col1 head} & \textbf{col2 head} & \textbf{col3 head} & \multicolumn{1}{@{}l@{}}{\textbf{col4 head}} & \textbf{col5 head} \\
\midrule
col1 text & col2 text & col3 text & 12.34 & col5 text 1 \\
col1 text & col2 text & col3 text & 1.62 & col5 text 2 \\
col1 text & col2 text & col3 text & 51.809 & col5 text 3 \\
\bottomrule
\end{tabular*}} % \\
{Unnumbered table footnotes.}
\end{table}

\begin{table}
\caption{This is sample table caption.\label{tab2}}
{\tabcolsep0pt\begin{tabular*}{\textwidth}{@{\extracolsep{\fill}}lcccc@{}}
\toprule
\textbf{col1 head} & \textbf{col2 head} & \textbf{col3 head} & \textbf{col4 head} & \textbf{col5 head} \\
\midrule
col1 text & col2 text & col3 text & col4 text & col5 text$^\dagger$ \\
col1 text & col2 text & col3 text & col4 text & col5 text \\
col1 text & col2 text & col3 text & col4 text & col5 text$^\ddagger$ \\
\bottomrule
\end{tabular*}} % \\
{$^\dagger$Example for a first table footnote. \\
$^\ddagger$Example for a second table footnote.}
\end{table}



\begin{sidewaystable}
\caption{Sideways table caption. For decimal alignment refer column 4 to 9 in tabular preamble.\label{tab3}}%
{\tabcolsep0pt\begin{tabular*}{\textheight}{@{\extracolsep\fill}lccd{2,4}d{2,4}d{2,4}d{2,4}d{2,4}d{2,4}@{\extracolsep\fill}}%
\toprule
\textbf{S No} & \textbf{col2 head} & \textbf{col3 head} & \multicolumn{1}{c}{\textbf{10}} &\multicolumn{1}{c}{\textbf{20}} &\multicolumn{1}{c}{\textbf{30}} &\multicolumn{1}{c}{\textbf{10}} &\multicolumn{1}{c}{\textbf{20}} &\multicolumn{1}{c}{\textbf{30}} \\
\midrule
1 &col2 text &col3 text &0.7568&1.0530&1.2642&0.9919&1.3541&1.6108 \\
2 & &col2 text &12.5701 &19.6603&25.6809&18.0689&28.4865&37.3011 \\
3 &col2 text & col3 text &0.7426&1.0393&1.2507&0.9095&1.2524&1.4958 \\
4 & &col3 text &12.8008&19.9620&26.0324&16.6347&26.0843&34.0765 \\
5 & col2 text & col3 text &0.7285&1.0257&1.2374&0.8195&1.1407&1.3691 \\
6 & & col3 text &13.0360&20.2690&26.3895&15.0812&23.4932&30.6060 \\
\bottomrule
\end{tabular*}}
\end{sidewaystable}

\begin{sidewaysfigure}
\centerline{\includegraphics[width=25pc,height=15pc]{Figure1.eps}}
\caption{Sideways figure caption. Sideways figure caption. Sideways figure caption. Sideways figure caption. Sideways figure caption. Sideways figure caption.\label{fig3}}
\end{sidewaysfigure}

\subsection{Examples for Enunciations}\label{sec4.7}

This section cases where the concentration gradient is the only cause for the flow of matter. We have seen that such situations can be studied using tiny amounts of trace elements in an otherwise homogeneous material.

\begin{theorem}[Theorem subhead]\label{thm1}
We have seen that such situations can be studied using tiny amounts of trace elements in an otherwise homogeneous material. However, from a general viewpoint a diffusion flux is proportional to the gradient of the chemical potential.
\end{theorem}

\begin{proposition}\label{prop1}
We have seen that such situations can be studied using tiny amounts of trace elements in an otherwise homogeneous material. However, from a general viewpoint a diffusion flux is proportional to the gradient of the chemical potential.
\end{proposition}

However, from a general viewpoint a diffusion flux is proportional to the gradient of the chemical potential, from a general viewpoint a diffusion flux is proportional to the gradient of the chemical potential.

\begin{proof}
Example for proof text. Example for proof text. Example for proof text. Example for proof text. Example for proof text. Example for proof text. Example for proof text. Example for proof text. Example for proof text. Example for proof text.
\end{proof}

\begin{definition}[Definition sub head]\label{def1}
We have seen that such situations can be studied using tiny amounts of trace elements in an otherwise homogeneous material. However, from a general viewpoint a diffusion flux is proportional to the gradient of the chemical potential.
\end{definition}

\begin{example}
We have seen that such situations can be studied using tiny amounts of trace elements in an otherwise homogeneous material. However, from a general viewpoint a diffusion flux is proportional to the gradient of the chemical potential.
\end{example}

\begin{proof}[Proof of Theorem~\ref{thm1}]
In $G$ denotes Gibbs free energy, $n_{i}$ the number of moles of species $i$, $T$ the temperature, and $p$ the hydrostatic pressure. The chemical potential depends on the alloy composition.
\end{proof}


\subsection{Code Display}\label{sec4.8}

Use \verb+\begin{verbatim}...\end{verbatim}+ for program codes without math. Use \verb+\begin{alltt}...+ \verb+program codes\end{alltt}+ for program codes with math. Based on the text provided inside the optional argument of \verb+\begin{code}[Psecode|Listing|Box|Code|Specification|Procedure|+ \verb+Sourcecode|Program]...+ \verb+\end{code}+ tag corresponding boxed like floats are generated. Also note that \verb+\begin{code}[Code|Listing]...+ \verb+\end{code}+ tag with either Code or Listing text as optional argument text are set with computer modern typewriter font. All other code environments are set with normal text font. Refer below example:

\begin{lstlisting}[caption={Descriptive Caption Text},label=DescriptiveLabel]
for i:=maxint to 0 do
 begin
 { do nothing }
 end;
Write('Case insensitive ');
WritE('Pascal keywords.');
\end{lstlisting}

\subsection{Algorithms}\label{sec4.9}
Below is Algorithm example. Use algorithm package documentation for more details:
\begin{algorithm}
\caption{Pseudocode for our algorithm}\label{alg1}
\begin{algorithmic}
 \For each frame
 \For water particles $f_{i}$
 \State compute fluid flow
 \State compute fluid--solid interaction
 \State apply adhesion and surface tension
 \EndFor
 \EndFor
\end{algorithmic}
\end{algorithm}

\subsection{Cross-referencing}\label{sec4.10}

The use of the \LaTeX\ cross-reference system for figures, tables, equations, etc., is encouraged (using \verb"\ref{<name>}" and \verb"\label{<name>}").

\subsection{Acknowledgements}\label{sec4.11}

An Acknowledgements section is started with \verb"\ack" or \verb"\acks" for \textit{Acknowledgement} or \textit{Acknowledgements}, respectively. It must be placed just before the References.

\subsection{Bibliography}\label{sec4.12}

The bibliography section is using the standard "natbib" package for author-year citation. The normal commands for producing the reference list are:
\begin{quote}
where \verb"\bibitem{x-ref label}" corresponds to \verb"\citet{x-ref label}" (direct citation) or \verb"\citep{x-ref label}" (indirect citation) in the body of the article.
\end{quote}

For those authors that are using \BibTeX, \textbf{wb\_stat.bst} is included in the zip package as well. This bibliography style will format the reference based on \textit{Stat} requirement (as seen in the reference section).

\subsection{Visuanimation}\label{sec4.13}

Visuanimation is a term coined for visualization through animations, please refer CTAN for \LaTeX\ packages. It requires the package \textit{animate} and \textit{multimedia}. It allows to embed animations in the paper itself and to store larger movies in the online supplemental material. Examples of statistics research projects using a variety of visuanimations range from exploratory data analysis of image data sets to spatio-temporal extreme event modelling, from multiscale analysis of classification methods to the study of the effects of a simulated explosive volcanic eruption and emulation of climate model output, from spatio-temporal wind roses to point processes on the sphere. The use of visuanimations in \textit{Stat} papers is highly encouraged.

\subsection{Copyright Statement}

Please be aware that the use of this \LaTeXe\ class file is governed by the following conditions.
Copyright \copyright\ 2018 John Wiley \& Sons, Ltd, The Atrium, Southern Gate, Chichester, West Sussex, PO19~8SQ, UK. All rights reserved.

\subsection{Rules of Use}

This class file is made available for use by authors who wish to prepare an article for publication in \textit{Stat} published by John Wiley \& Sons, Ltd. The user may not exploit any part of the class file commercially.

This class file is provided on an \emph{as is} basis, without warranties of any kind, either express or implied, including but not limited to warranties of title, or implied warranties of merchantablility or fitness for a particular purpose. There will be no duty on the author[s] of the software or John Wiley \& Sons, Ltd to correct any errors or defects in the software. Any statutory rights you may have remain unaffected by your acceptance of these rules of use.

\subsection{The Three Golden Rules}

Before we proceed, we would like to stress \emph{three golden rules} that need to be followed to enable the most efficient use of your code at the typesetting stage:
\begin{enumerate}
\item[(i)] keep your own macros to an absolute minimum;

\item[(ii)] as \TeX\ is designed to make sensible spacing decisions by itself, do \emph{not} use explicit horizontal or vertical spacing commands, except in a few accepted (mostly mathematical) situations, such as \verb"\," before a differential~d, or \verb"\quad" to separate an equation from its qualifier;

\item[(iii)] follow this sample for styles
\end{enumerate}


\ack{I would like to thank....}

%% \acks{I would like to thank....}

\appendix

\section{Appendix Examples}\label{appsec1}

A researcher must decide which observed variables to include in the theoretical model and how these observed variables measure latent variables.
\begin{align}\label{App1}
 \|\tilde{X}(k)\|^2 &=\frac{\left\|\sum\limits_{i=1}^{p}\tilde{Y}_i(k)+\sum\limits_{j=1}^{q}\tilde{Z}_j(k) \right\|^2}{(p+q)^2} \\
 \|\tilde{Y}(k)\|^2 &= \frac{\sum\limits_{i=1}^{p}\left\|\tilde{Y}_i(k)\right\|^2+\sum\limits_{j=1}^{q}\left\|\tilde{Z}_j(k)\right\|^2 }{p+q}.
\end{align}

\begin{table}[ht]%
\caption{This is sample appendix table caption.\label{apptab1}}
{\tabcolsep0pt\begin{tabular*}{\textwidth}{@{\extracolsep{\fill}}lccd{2,3}c@{}}
\toprule
\textbf{col1 head} & \textbf{col2 head} & \textbf{col3 head} & \multicolumn{1}{@{}l@{}}{\textbf{col4 head}} & \textbf{col5 head} \\
\midrule
col1 text & col2 text & col3 text & 12.34 & col5 text 1 \\
col1 text & col2 text & col3 text & 1.62 & col5 text 2 \\
col1 text & col2 text & col3 text & 51.809 & col5 text 3 \\
\bottomrule
\end{tabular*}} % \\
{Unnumbered table footnotes.}
\end{table}

\section{Theoretical Model}\label{appsec2}

A researcher must decide which observed variables to include in the theoretical model and how these observed variables measure latent variables.
\begin{equation}\label{App2}
D(\alpha_{1}, \alpha_{2}, \alpha_{3})=\alpha_{1}^{2}D_{I}+\alpha_{2}^{2}D_{II}+\alpha_{3}^{2}D_{III},
\end{equation}
Although there will be a gradient in the concentration of the trace element, its total concentration can be kept so small that the overall composition of the sample during the investigation does practically not change.

\begin{figure}[!h] %% Appendix Figure 1
\centerline{\includegraphics[height=10pc]{Figure1.eps}}
\caption{A short description of appendix figure content should go here.\label{appF1}}
\end{figure}


%% In using BibTeX, use wb_stat.bst
%\bibliographystyle{wb_stat}
%\bibliography{WileySTAT}

\section*{Related Articles}

\textbf{Scoring Rules; Subjective Probabilities: Theory; Subjective Randomness; Subjective Probability
and Human Judgement; Subjective Probabilities: Overview; Subjective Expected Utility; Expert
Opinion in Reliability; Expert Judgment; Expert Elicitation for Risk Assessment; Prior Distribution
Elicitation.}


% Non-BibTeX users, below can be used (copy of BibTeX *.bbl file )
\begin{thebibliography}{3}
\newcommand{\enquote}[1]{`#1'}
\providecommand{\natexlab}[1]{#1}
\expandafter\ifx\csname urlstyle\endcsname\relax
 \providecommand{\doi}[1]{doi:\discretionary{}{}{}#1}\else
 \providecommand{\doi}{doi:\discretionary{}{}{}\begingroup
 \urlstyle{rm}\Url}\fi

\bibitem[{Benson(1992)}]{R2}
Benson, D (1992), \enquote{Computational methods in {L}agrangian and {E}ulerian
 hydrocodes,} \emph{Comput {M}ethod {A}ppl {M}}, \textbf{99}(2--3), pp.
 235--394.

\bibitem[{Dukowicz(1984)}]{R3}
Dukowicz, J (1984), \enquote{Conservative rezoning (remapping) for general
 quadrilateral meshes,} \emph{J {C}omput {P}hys}, \textbf{54}(3), pp.
 411--424.

\bibitem[{Hirt et~al.(1974)Hirt, Amsden \& Cook}]{R1}
Hirt, C, Amsden, A \& Cook, J (1974), \enquote{An arbitrary
 {L}agrangian-{E}ulerian computing method for all flow speeds,} \emph{J
 {C}omput {P}hys}, \textbf{14}(3), pp. 227--253.

\end{thebibliography}


\subsection*{Further Reading}
\begin{enumerate}
\item Knuth, D.~E. (1986). \textit{The {\TeX}book}. Reading, MA: Addison-Wesley.

\item Lamport~L. 1994. LaTeX is document preparation system, commonly used by scientists engineers, mathematicians, and professionals. \textit{\LaTeX: A Document Preparation System} (2nd~edn). Addison-Wesley.

\item Mittelbach~F, Goossens~M. 2004. \textit{The \LaTeX\ Companion} (2nd~edn). Addison-Wesley.
\end{enumerate}

\end{document}

© 2021 Overleaf Deutsch|Privacy and TermsSicherheitKontaktiere unsÜber unsBlogOverleaf on TwitterOverleaf on FacebookOverleaf on LinkedIn
